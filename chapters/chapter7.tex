\section{Wnioski} \label{wnioski}
Głównym celem pracy było stworzenie projektu aplikacji, która umożliwi bezpieczne poruszanie się po polskich górach Tatrach. Cel ten został osiągnięty po uprzednim zapoznaniu się z problematyką wyznaczania tras na podstawie obecnej lokalizacji użytkownika oraz powiadomienia go o zagrożeniach występujących w okolicy. Wszystkie założenia zostały dokładnie przeanalizowane tak, aby aplikacja mogła działać, a użytkownik bezpiecznie zdobywał polskie Tatry.

Użyte technologie spełniły oczekiwania i pozwoliły na stworzenie sprawnie działającej aplikacji. Wykorzystanie usług oferowanych przez Firebase znacznie usprawniło zaprogramowanie aplikacji, gdyż dzięki nim zyskała ona serwer, a także bazę danych przechowującą m.in. informacje o użytkownikach oraz zgłoszeniach. Zaimplementowanie API Google Maps pozwoliło na sprawne dodawanie szlaków przez administratora.
Podczas tworzenia aplikacji napotkano kilka problemów, które po wielu próbach zostały rozwiązane. Największym z nich było zmodyfikowanie szlaków na mapie, aby po kliknięciu na nie system wyświetlał o nich informacje. Funkcja ta została zastąpiona listą szlaków, z której użytkownik może wybrać ten, który go interesuje.

Projekt aplikacji zakładał, że każdy szlak będzie zawierał informacje o dostępnych w pobliżu atrakcjach turystycznych. Przez to, że napotkano problem związany z klikanym wybieraniem szlaku z mapy i wyświetleniem informacji o nim, opisy atrakcji turystycznych musiały zejść na drugi plan.

Aplikacja “Harnasie” posiada potencjał do dalszego rozwoju funkcjonalności przez nią oferowanych. Obszar szlaków może być poszerzony o inne rejony gór, dzięki czemu użytkownik będzie mógł korzystać z aplikacji nie tylko w Tatrach. Baza danych może być także uzupełniona o nową tabelę szlaki. Administrator mógłby wtedy nimi zarządzać, np. włączać i wyłączać z użycia. Do widoku map można dodać także wszelakie atrakcje, które turyści mogą spotkać na szlaku wraz z ich opisami i możliwością dodawania do opisów własnych zdjęć.

