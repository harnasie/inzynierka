\section{Cel i zakres pracy}

    Celem pracy jest wykonanie projektu i implementacja w języku Java aplikacji mobilnej na urządzenia z systemem Android wspomagającej jednodniowe górskie wędrówki w Tatrach Polskich. Aplikacja wykorzystuje moduł GPS telefonu, mapy Google oraz usługi oferowane przez Firebase.


    \vspace{0.5cm}
    
    \textbf{Zakres działań obejmuje:}
    \begin{itemize}
        \item przegląd rynku w poszukiwaniu aplikacji oferujących mobilne mapy rejonu Karpat
        \item sformułowanie wymagań funkcjonalnych oraz niefunkcjonalnych, 
        \item stworzenie diagramów przypadków użycia oraz scenariuszy,  
        \item utworzenie konceptualnego diagramu klas, 
        \item wykonanie projektu modelu danych,
        \item wybór narzędzi projektowych i programistycznych, 
        \item projekt interfejsu użytkownika i administratora,
        \item implementację aplikacji,
        \item testy automatyczne i manualne
    \end{itemize}
    \textbf{Zakres implementacji obejmuje:}
    \begin{itemize}
        \item zaprojektowanie i implementacja bazy danych,
        \item umożliwienie planowania trasy,
        \item nawigację po wybranej trasie,
        \item stworzenie formularza do zgłaszania zagrożeń i zapisywanie wysłanych zagrożeń,
        \item wysyłanie powiadomień ogólnosystemowych do wszystkich użytkowników odnośnie zagrożeń bądź wyłączenia szlaków z użycia i remontów,
        \item autoryzację i logowanie użytkowników,
        \item zapisywanie historii wędrówek i przedstawianie jej w formie wykresów lub raportów,
        \item dostęp z poziomu aplikacji do numerów alarmowych.
    \end{itemize}
    \subsection{Podział pracy}
    \begin{itemize}
        \item Streszczenie i wstęp - Katarzyna Jakubowska
        \item Cel i zakres pracy - wspólnie
        \item Analiza rynku - wspólnie
        \item Przegląd technologii - Katarzyna Jakubowska
        \item Projekt aplikacji - wspólnie
        \item Implementacja
        \begin{itemize}[label=$\circ$]
            \item Mapa - Magdalena Kamińska
            \begin{itemize}
                \item wyznaczanie trasy z punktami postojowymi (tworzenie url) 
                \item obsługa markerów oznaczających schroniska i szczyty (widoczność na mapie)
                \item pobieranie punktów współrzędnych z plików .kml i rysowanie szlaków na mapie 
                \item obsługa nawigacji po trasie za pomocą aplikacji Google Maps 
                \item wybór szlaku z listy i pokazanie go na mapie 
            \end{itemize}
            \item Firebase
            \begin{itemize}
                \item integracja funkcji oferowanych przez Firebase z aplikacją: bazy danych Firebase Firestore, Firebase Authentication - Katarzyna Jakubowska,
                \item zapis i odczyt zagrożeń z Firebase Firestore - Magdalena Kamińska,
                \item obsługa akceptacji zgłoszeń i ustawienie markera z zagrożeniem na mapie - Magdalena Kamińska,
                \item utworzenie wykresu aktywności użytkownika na podstawie danych pobranych z Firebase - Magdalena Kamińska,
                \item pobieranie z Firebase Storage i zapis lokalny na urządzeniu plików .kml ze szlakami i szczytami - Magdalena Kamińska,
                \item logowanie i autoryzacja użytkowników z podziałem na role - Katarzyna Jakubowska,
                \item zapis aktywności użytkownika do bazy danych Firestore - Katarzyna Jakubowska
            \end{itemize}
            \item Powiadomienia o zagrożeniach
            \begin{itemize}
                \item wysyłanie powiadomień o zagrożeniach lokalnych do użytkowników znajdujących się w pobliżu - Katarzyna Jakubowska
                \item wysyłanie powiadomień ogólnosystemowych do wszystkich użytkowników z treścią komunikatów TOPR, TPN oraz GOPR o zagrożeniach występujących w Tatrach - Katarzyna Jakubowska
            \end{itemize}
        \end{itemize}
        \item Interfejsy aplikacji
        \begin{itemize}[label=$\circ$]
            \item interfejs administratora - Katarzyna Jakubowska,
            \item interfejs użytkownika - Magdalena Kamińska
        \end{itemize}
        \item Testy manualne i jednostkowe - Katarzyna Jakubowska
        \item Wnioski - wspólnie
        \end{itemize}

    \textbf{Implementacja - opisy}
    \begin{itemize}
        \item Moduł map - Magdalena Kamińska
        \item Moduł powiadomień - Katarzyna Jakubowska
        \item Moduł zgłaszania zagrożeń 
        \item Interfejs użytkownika - Magdalena Kamińska
        \item Interfejs administratora - Katarzyna Jakubowska
    \end{itemize}

    \newpage