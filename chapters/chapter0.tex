\section{Streszczenie}
Niniejsza praca jest opisem procesu projektowania i implementacji aplikacji mobilnej usprawniającej nawigowanie po szlakach górskich. Wykorzystuje do tego mapy udostępnione do użytku przez Google Maps Platform, a cała aplikacja napisana została w języku Java. Dostęp do funkcji aplikacji ograniczony jest odpowiednimi rolami: administrator oraz użytkownik. Administrator jako kluczowa jednostka w działaniu systemu zarządza szlakami, wysyła ogólnosystemowe powiadomienia, zatwierdza zgłoszenia wysłane przez użytkowników. Użytkownik natomiast może cieszyć się dostępem do nawigacji, przeglądania map celem wybrania trasy, dostępem do swojej historii wędrówek oraz skrótem do telefonów alarmowych przydatnych w razie wypadku na szlaku. \\
\textbf{Słowa klucze:} Java, aplikacja mobilna, nawigacja, mapy, wyznaczanie trasy, Android

\section*{Abstract}
This thesis describes the design and implementation of a mobile application that improves navigation on mountain trails. For this purpose, it uses maps made available for use by Google Maps Platform, and the entire application was written in Java. Access to application functions is limited by appropriate roles: administrator and user. The administrator, as a key unit in the operation of the system, manages trails and attractions, sends system-wide notifications, and approves applications sent by users. The user can enjoy access to navigation, browsing maps to choose a route, access to their hiking history and easier access to emergency phones useful in the event of an accident on the trail. \\
\textbf{Keywords:} Java, mobile application, navigation, maps, route mapping, Android